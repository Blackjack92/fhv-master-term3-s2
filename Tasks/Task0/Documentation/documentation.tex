\documentclass[
a4paper,     %% defines the paper size: a4paper (default), a5paper, letterpaper, ...
headsepline, %% add a horizontal line below the column title
fleqn,       %% equation left-justified (instead of centered)
12pt         %% set default font size to 12 point
]{scrartcl}  %% article, see KOMA documentation (scrguide.dvi)

%%%%%%%%%%%%%%%%%%%%%%%%%%%%%%%%%%%%%%%%%%%%%%%%%%%%%%%%%%%%%%%%%%%%%%%%%%%%%%%%
%%%
%%% packages
%%%

%%%
%%% encoding and language set
%%%

%%% ngerman: language set to new-german
\usepackage{ngerman}

%%% babel: language set (can cause some conflicts with package ngerman)
%%%        use it only for multi-language documents or non-german ones
%\usepackage[ngerman]{babel}

%%% inputenc: coding of german special characters
\usepackage[latin1]{inputenc}

%%% fontenc, ae, aecompl: coding of characters in PDF documents
\usepackage[T1]{fontenc}
\usepackage{ae,aecompl}

%%% Graphic stuff
\usepackage{graphicx}

%%%
%%% technical packages
%%%

%%% amsmath, amssymb, amstext: support for mathematics
\usepackage{amsmath,amssymb,amstext}

%%% psfrag: replace PostScript fonts
%\usepackage{psfrag}

%%% listings: include programming code
%\usepackage{listings}

%%% units: technical units
\usepackage{units}

%%%
%%% layout
%%%

%%% scrpage2: KOMA heading and footer
%%% Note: if you don't use this package, please remove 
%%%       \pagestyle{scrheadings} and corresponding settings
%%%       below too.
\usepackage{scrpage2}

%%%%%%%%%%%%%%%%%%%%%%%%%%%%%%%%%%%%%%%%%%%%%%%%%%%%%%%%%%%%%%%%%%%%%%%%%%%%%%%%
%%%
%%% user defined commands
%%%

%%% \mygraphics{}{}{}
%% usage:   \mygraphics{width}{filename_without_extension}{caption}
%% example: \mygraphics{0.7\textwidth}{rolling_grandma}{This is my grandmother on inlinescates}
%% provides: including centered pictures/graphics with a boldfaced caption below
%% 
\newcommand{\mygraphics}[3]{
  \begin{center}
    \includegraphics[width=#1, keepaspectratio=true]{#2} \\
    \textbf{#3}
  \end{center}
}

%%%%%%%%%%%%%%%%%%%%%%%%%%%%%%%%%%%%%%%%%%%%%%%%%%%%%%%%%%%%%%%%%%%%%%%%%%%%%%%%
%%%
%%% set heading and footer
%%%

%%% scrheadings default: 
%%%      footer - middle: page number
\pagestyle{scrheadings}

%%% heading - left
 \ihead[]{Kevin Wallis}

%%% heading - center
% \chead[]{}

%%% heading - right
 \ohead[]{Aufgabe 0}

%%% footer - left
% \ifoot[]{}

%%% footer - center
% \cfoot[]{}

%%% footer - right
% \ofoot[]{}



%%%%%%%%%%%%%%%%%%%%%%%%%%%%%%%%%%%%%%%%%%%%%%%%%%%%%%%%%%%%%%%%%%%%%%%%%%%%%%%%
%%%
%%% begin document
%%%

\begin{document}

 \pagenumbering{roman} %% small roman page numbers
 \pagenumbering{arabic} %% normal page numbers (include it, if roman was used above)

%%%%%%%%%%%%%%%%%%%%%%%%%%%%%%%%%%%%%%%%%%%%%%%%%%%%%%%%%%%%%%%%%%%%%%%%%%%%%%%%
%%%
%%% begin main document
%%% structure: \section \subsection \subsubsection \paragraph \subparagraph
%%%

\section{Einleitung}
\label{sec:einleitung}

Ziel dieser Arbeit ist es weniger, eine Antriebswelle auf ihre Belastung durchzurechnen, als vielmehr ein Beispiel zu bringen, wie ein (technischer) Projektbericht mit der Vorlage \textsf{template\_bericht.tex} verfasst werden kann. Es wird darauf Wert gelegt, da� die h�ufigsten Elemente wie Tabellen, Abbildungen und mathematische Umgebungen nicht fehlen.


\section{Berechnung}
\label{sec:berechnung}

Unsere Aufgabe ist es, eine statisch bestimmt gelagerte Welle zu untersuchen.

\subsection{Wellenlagerung}
\label{sec:wellenlagerung}

Der schematische Aufbau ist in Abb.~\ref{fig:schema} ersichtlich. Der Motor (links) treibt die Welle mit vorgegebenem Drehmoment an; hier sei vorausgesetzt, da� der Durchmesser des Bauteils ausreicht, um das Torsionsmoment aufnehmen zu k�nnen. Weiters soll durch eine spezielle Lagerung der Motor keine vertikalen Kr�fte auf die Welle �bertragen k�nnen. An der Schleifscheibe (rechts) k�nnen jedoch Querkr�fte auftreten, die zu einer zus�tzlichen Belastung der Welle f�hren. Untersucht werden sollen die Kr�fte in den Lagern. 

%%%
%%% end main document
%%%
%%%%%%%%%%%%%%%%%%%%%%%%%%%%%%%%%%%%%%%%%%%%%%%%%%%%%%%%%%%%%%%%%%%%%%%%%%%%%%%%

 \appendix  %% include it, if something (bibliography, index, ...) follows below
 \bibliographystyle{plain}

%%% name of the bibliography file
 \bibliography{projekt.bib}

\end{document}